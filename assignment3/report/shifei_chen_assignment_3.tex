%%%%%%%%%%%%%%%%%%%%%%%%%%%%%%%%%%%%%%%%%%%%%%%%%%%%%%%
%%% LATEX FORMATTING - LEAVE AS IS %%%%%%%%%%%%%%%%%%%%
\documentclass[11pt]{article} % documenttype: article
\usepackage[top=20mm,left=20mm,right=20mm,bottom=15mm,headsep=15pt,footskip=15pt,a4paper]{geometry} % customize margins
\usepackage{times} % fonttype
\usepackage{listings}

\makeatletter         
\def\@maketitle{   % custom maketitle 
\begin{center}
{\bfseries \@title}
{\bfseries \@author}
\end{center}
\smallskip \hrule \bigskip }

%%%%%%%%%%%%%%%%%%%%%%%%%%%%%%%%%%%%%%%%%%%%%%%%%%%%%%%%%%%%%%%%%%%%
%%% MAKE CHANGES HERE %%%%%%%%%%%%%%%%%%%%%%%%%%%%%%%%%%%%%%%%%%%%%%
\title{{\LARGE Syntactic Parsing: Assignment 3}\\[1.5mm]} % Replace 'X' by number of Assignment
\author{Shifei Chen} % Replace 'Firstname Lastname' by your name.

%%%%%%%%%%%%%%%%%%%%%%%%%%%%%%%%%%%%%%%%%%%%%%%%%%%%%%%%%%%%%%%%%%%%
%%% BEGIN DOCUMENT %%%%%%%%%%%%%%%%%%%%%%%%%%%%%%%%%%%%%%%%%%%%%%%%%
%%% From here on, edit document. Use sections, subsections, etc.
%%% to structure your answers.
\begin{document}
\maketitle

\section{Arc-eager Oracle Error Analysis}

As stated in the instruction, ``Ideally, the trees output by your oracle parser should be identical to the original trees from the treebank...'', but the reality is that there are some errors from my arc-eager oracle. Thinking of this example,

\begin{lstlisting}
I           PRON    5	nsubj       I           PRON    5   nsubj
was         VERB    5	cop         was         VERB    5   cop
on          ADP     5	case        on          ADP     5   case
my          PRON    5	nmod:poss   my          PRON    5   nmod:poss
way         NOUN    0	root        way         NOUN    0   root
to          ADP     8	case        to          ADP     8   case
my          PRON    8	nmod:poss   my          PRON    8   nmod:poss
wedding     NOUN    5	nmod        wedding     NOUN    5   nmod
fearing     VERB    1	acl         fearing     VERB    0   root
death       NOUN    9	dobj        death       NOUN    9   dobj
,           PUNCT   9	punct       ,           PUNCT   9   punct
basically   ADV     9	advmod      basically   ADV     9   advmod
.           PUNCT   5	punct       .           PUNCT   0   root
"           PUNCT   5	punct       "           PUNCT   0   root    
\end{lstlisting}

One the left hand side we have the dependency tree from the golden standard, while on the right hand side it's the tree predicted from my oracle. The word ``fearing'' and the last two punctuation, ``.'' and the quotation mark have the incorrect dependencies and labels.

In general, errors happen in a non-projective tree, when a word has a long arc to a previous word that has already been removed from the stack. In the case above, when the oracle was trying to build a left arc for the word ``fearing'', it's configuration at that moment looked like this,

\begin{lstlisting}
BUFFER
fearing | death | , | basically | . | "
STACK
ROOT | wedding
\end{lstlisting}

The head word of ``fearing'', ``I'' was already been removed when building the left arc (way, nsubj, I) hence at this moment it was not possible to find the head word ``I'' in the stack again. Also, in an ideal case the head word should stay in the stack if there are dependents later in the sentence, however as the program was predicting and parsing the sentence rule by rule at the same time, the \verb|trasition()| function can never be able to know how many dependents are still remain for a specific head word.

\section{Arc-Standard Parser}

The non-projective problem went even worse when I was implementing the arc-standard parser as it cannot handle that kind of sentences at all. As a workaround I had to raise an exception and warn the user in the \verb|transition()| function.

Another thing is when building a right arc in the arc-standard parser, we need to loop through all the existing arcs to find if there's any non-implemented arc $(S_1, l, k)$ for the top word in the stack $S_1$.

I've also rewriten the argument parsing part in \verb|transition.py| so now it can parse sentences in both the arc-eager and arc-standard way (enabled by the argument \verb|-s|).

\end{document}